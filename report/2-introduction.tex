
\begin{savequote}[45mm]
I love scotch. Scotchy, scotch, scotch. Here it goes down, down into my belly... 
\qauthor{Ron Burgundy, \textit{Anchorman}}
\end{savequote}

\chapter{Introduction}

It's easy to break a budget on alcohol, as many college students know. But it becomes exponentially easier with scotch whisky\footnote{Scotch whisky is spelled without the "e"} with the rarest of breeds fetching upwards of \$460,000.\cite{Macallan64} But even middling bottles can easily run over \$100, leaving many to wonder why anyone would spend their hard earned money on something that is sure to have the same net result as a \$10 bottle of cheap swill that can be mixed with cola.

Scotch drinkers tend to think about this differently. For many, scotch is a hobby, and collecting and sampling new whiskies is the goal, as opposed to just drinking. Many people appreciate the arduous time spent producing the whisky, from the distillation to the aging. Scotch takes much longer to age than bourbon, due to the climate in Scotland, and produces a vastly different flavor profile. The standard minimum age for scotch is typically twelve years, though due to rising demands, many distilleries have been experimenting with younger batches, letting some rest as little as five years. Scotch must be aged for three years by law, but for most scotches, desirability increases with the age.

In this paper, we examine the prices of scotch and how it correlates to age. We will also look at various online retailers and compare them with each other as well as local retailers in the Omaha, Nebraska area and examine the variance of whisky prices.


\section{Background and Recent Research}
\subsection{<any sub section here>}

\subsection{Literature Survey}

\subsubsection{<Sub-subsection title>}
some text\cite{citation-1-name-here}, some more text

\subsubsection{<Sub-subsection title>}
even more text\footnote{<footnote here>}, and even more.

\section{Motivation}

